\documentclass{article}
\usepackage{amsmath}
\usepackage{float}
\usepackage{fvextra} 
\usepackage{datetime}
\title{Mechine Learning - Assignment 3}
\author{Michael Dadush \\ 206917908 \and Shay Gali \\ 315202242}
\date{\monthname[\month] \ \the\year}

\begin{document}
\maketitle

\section{Analysis of KNN Parameters and Results}

Based on our experiment results with different k-nearest neighbor parameters, we can make important observations about the classifier performance. We will analyze the results for different k and p values.


\subsection{Code Output}
This is the code output for different k and p values:
\begin{Verbatim}[fontsize=\small, xleftmargin=0pt, xrightmargin=0pt]
    round       | Average Empirical Errors  |  Average True Errors  |  Error Differences
p = 1   k = 1   |         0.014600          |       0.130800        |      0.116200
p = 1   k = 3   |         0.055800          |       0.093800        |      0.038000
p = 1   k = 5   |         0.060400          |       0.090000        |      0.029600
p = 1   k = 7   |         0.063800          |       0.090000        |      0.026200
p = 1   k = 9   |         0.065400          |       0.090000        |      0.024600
                |                           |                       |
p = 2   k = 1   |         0.014800          |       0.126200        |      0.111400
p = 2   k = 3   |         0.054400          |       0.092000        |      0.037600
p = 2   k = 5   |         0.062800          |       0.085400        |      0.022600
p = 2   k = 7   |         0.064200          |       0.085200        |      0.021000
p = 2   k = 9   |         0.063600          |       0.084400        |      0.020800
                |                           |                       |
p = inf k = 1   |         0.014800          |       0.136200        |      0.121400
p = inf k = 3   |         0.056600          |       0.090400        |      0.033800
p = inf k = 5   |         0.062400          |       0.085200        |      0.022800
p = inf k = 7   |         0.063000          |       0.086400        |      0.023400
p = inf k = 9   |         0.065400          |       0.084200        |      0.018800
\end{Verbatim}


\subsection{Best Parameters}
From looking at the data results, we found that the best parameters are:
\begin{itemize}
    \item p = 2 with k = 9 (error rate = 0.0844 or 8.44\%)
    \item Also good is p = $\infty$ with k = 9 (error rate = 0.0842 or 8.42\%)
\end{itemize}

These parameters give us the lowest Average True Errors, which is most important for real performance.

\subsection{Analysis of k Parameter}
When we look at how k affects the results:
\begin{itemize}
    \item k = 1 gives worst performance on test data for all p values
    \item When k gets bigger, the true error usually gets smaller
    \item This shows us that using more neighbors helps make better predictions
\end{itemize}

\subsection{Analysis of p Parameter}
For the distance metric p:
\begin{itemize}
    \item p = 1 does not work as good as p = 2 or p = $\infty$
    \item p = 2 (Euclidean) and p = $\infty$ work almost same, but p = 2 is little better
    \item We think this shows that using regular distance (p = 2) works better for our data than other options
\end{itemize}

\subsection{Overfitting Analysis}
We see clear overfitting in our results, especially with k = 1:

For k = 1:
\begin{itemize}
    \item Very small empirical errors ($\approx$ 0.014-0.015)
    \item Much bigger true errors ($\approx$ 0.126-0.136)
    \item Big differences between errors ($\approx$ 0.111-0.121)
\end{itemize}

When k gets bigger:
\begin{itemize}
    \item Empirical errors get little bigger
    \item True errors get much smaller
    \item Differences between errors get smaller too
\end{itemize}

This pattern shows classic overfitting when k = 1. The model learns training data too well (small empirical error) but cannot work good on new data (big true error). Using bigger k helps fix this by taking average of more neighbors.

\subsection{Why These Parameters Work Best}
k = 9 with p = 2 gives best results because:
\begin{itemize}
    \item It uses enough neighbors to make predictions more stable
    \item It reduces overfitting (smallest difference between empirical and true errors)
    \item Euclidean distance (p = 2) seems to work better for relationships between features than Manhattan (p = 1) or maximum distance (p = $\infty$).
    \item Euclidean distance represents how the human eye sees distances, which is good for our data
    \item The big k value (k = 9) tells us there is probably lot of noise in data that needs to be averaged
\end{itemize}

\subsection{Conclusion}
Our analysis shows that k = 9 and p = 2 give best performance for this classification task. These parameters help balance between learning from data and not overfitting. The results also show importance of choosing right k value to avoid overfitting, especially avoiding k = 1 which memorizes training data too much.

\end{document}